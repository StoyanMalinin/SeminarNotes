\documentclass[12pt]{article}
\usepackage{lingmacros}
\usepackage{tree-dvips}
\usepackage[utf8]{inputenc}
\usepackage[russian]{babel}
\usepackage{amsmath,amssymb}
\usepackage{multirow}
\usepackage{hyperref}
\usepackage{caption}
\usepackage{tabularx}
\usepackage{enumitem}

\begin{document}

\section*{Дефиниции}
\subsection*{Greedy алгоритъм за оцветяване}
Нека имаме $G = <V, E>$. Номерираме върховете по някакъв си начин $v_1, v_2, ..., v_{n}$. И започвме да оцветяваме да оцветяваме върховете в този ред. Като оцветяваме един връх разглеждаме всичките му съседи с по-малки индекси и оцветяваме нашия връх с най-малкия цвят, който не е зает. Приемаме, че за цветове използваме естествените числа. Може да се види, че този алгоритъм се представя различно в зависимост от избраната първоначално наредба.

\section*{Задачи}

\subsection*{Лесни} 
\subsubsection*{Задача 1.1}
Да се докаже, че greedy алгоритъма може да се справи със $\chi(G)$ цвята при подходяща наредба.
\subsubsection*{Задача 1.2 - Diestel}
Да се покаже, че за всяко $n > 1$ съществува двуделен граф с $2n$ върха така, че при подходяща наредба greedy алгоритъма да използва $n$ цвята вместо $2$.
\subsubsection*{Задача 1.3 - Diestel}
Да се докаже, че следните две неща са еквивалентнни:
\begin{enumerate}
    \item $\chi(G) \leq k$
    \item Ребрата на $G$ може да се ориентират по такъв начин, че да няма цикли и да няма пътища с дължина $k$.
\end{enumerate}
\subsubsection*{Задача 1.4 - Astea}
Нека ребрата на $K_6$ са оцветени в черно и червено по произволен начин. Да се докаже, че съществува монохромен триъгълник.
\subsubsection*{Задача 1.5 - Записки по графи}
Да се докаже, че $\chi(G) \alpha(G) \geq n$.
\subsubsection*{Задача 1.6 - Записки по графи}
Да се докаже, че $\chi(G) \chi(\overline{G}) \geq n$.

\subsection*{По-забавни}
\subsubsection*{Задача 2.1 - Домашно - Информатика - 2021/2022}
Да се намери минималното оцветяване на ребрата на $K_n$.


\section*{Решения}

\end{document}