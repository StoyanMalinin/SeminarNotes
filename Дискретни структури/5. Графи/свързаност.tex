\documentclass[12pt]{article}
\usepackage{lingmacros}
\usepackage{tree-dvips}
\usepackage[utf8]{inputenc}
\usepackage[russian]{babel}
\usepackage{amsmath,amssymb}
\usepackage{multirow}
\usepackage{hyperref}
\usepackage{caption}
\usepackage{tabularx}
\usepackage{enumitem}

\begin{document}

\section*{Дефиниции}

\section*{Задачи}

\subsection*{Лесни}
\subsubsection*{Задача 1.1 - Kenneth Rosen}
Нека $G = (V, E)$ е граф. Да се докаже, че едно ребро е мост тогава и само тогава когато не е част от цикъл в графа. 
\subsubsection*{Задача 1.2 - Записки на Ангел Димитриев}
Нека $G = (V, E)$ е граф. Да се докаже, че $G$ е свързан или $\overline{G}$ е свързан. 
\subsubsection*{Задача 1.3}
Да се докаже, че ако $|E| < |V| - 1$, то $G$ не е свързан.

\subsection*{По-забавни}
\subsubsection*{Задача 2.1 - Kenneth Rosen}
Фермер трябва да закара вълк, овца и зеле през реката. Фермера има само една малка лодка, която може да побере само него и още един обект. Той може да преминава реката колкото си иска пъти. Проблемът е, че вълкът иска да изяде овцата, а овцата иска да изяде зелето. Ако фермерът е на същия бряг, където може да стане беля - няма проблем, но в противен случай тя е гарантирана. Състояние може да се представи като наредена двойка от вика $<FG, WC>$(фермерът и вълка са на левия край, а овцата и зелето на десния) или $<FWGC, \emptyset>$ (всички са на левия край).

\begin{itemize}
    \item Пресметнете и намерете всички разрешени състояния
    \item Постройте граф(насочен или ненасочен?), в който всеки връх е състояние, а ребрата са дали може да се премине от едното състояние до другото
    \item Намерете няколко пътя от началното $<FWGC, \emptyset>$ до финалното $<\emptyset, FWGC>$.
    \item Намерете най-оптималното решение (най-малко преминавания през реката).
\end{itemize}
\subsubsection*{Задача 2.2 - Kenneth Rosen}
Нека $G = (V, E)$ е граф. Да се докаже, че връх $v \in V$, който е край на мост $(u, v)$ е артикулационна точка тогава и само тогава, когато $d(v) \neq 1$. 
\subsubsection*{Задача 2.3 - Kenneth Rosen}
Нека $G = (V, E)$ е граф и $|V| \geq 2$. Да се докаже, че съществуват поне два върха, които не са артикулационни точки. 


\section*{Решения}

\end{document}