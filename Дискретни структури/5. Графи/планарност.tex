\documentclass[12pt]{article}
\usepackage{lingmacros}
\usepackage{tree-dvips}
\usepackage[utf8]{inputenc}
\usepackage[russian]{babel}
\usepackage{amsmath,amssymb}
\usepackage{multirow}
\usepackage{hyperref}
\usepackage{caption}
\usepackage{tabularx}
\usepackage{enumitem}

\begin{document}

\section*{Дефиниции}

\section*{Задачи}

\subsection*{Лесни}
\subsubsection*{Задача 1.1 - Kenneth Rosen}
Нека свързан планарен граф има 30 ребра. Ако някакво планарно вписване този граф разделя равнината на 20 региона, то колко върха има графът?
\subsubsection*{Задача 1.2 - Ralph P. Grimaldi}
Нека $G = (V, E)$ е граф. Нека някое планарно вписване на $G$ има 53 лица и всяко лице има поне 5 ребра на контура си. Докажете, че $V \geq 82$.
\subsection*{Задача 1.3 - Ralp P. Grimaldi}
Нека $G = (V, E)$ е граф. Нека $|V| \geq 11$. Докажете, че $G$ или $\overline{G}$ не е планарен.

\subsection*{По-забавни}
\subsection*{Задача 2.1 - Изпит - КН - 2021}
Докажете, че във всеки планарен граф има връх от степен, не по-голяма от 5. Има лесно доказателство с допускане на противното.
\begin{itemize}
    \item Първо напишете ясно и прецизно противното твърдение.
    \item Какво следва за броя на ребрата от противното твърдение? Открийте противоречие между това и нещо, изучавано на лекции.
\end{itemize}
Докажете по индукция по броя на върховете, че $\chi(G) \leq 6$ за всеки планарен граф $G$.

\section*{Решения}

\end{document}