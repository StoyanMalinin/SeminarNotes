\documentclass[12pt]{article}
\usepackage{lingmacros}
\usepackage{tree-dvips}
\usepackage[utf8]{inputenc}
\usepackage[russian]{babel}
\usepackage{amsmath,amssymb}
\usepackage{multirow}
\usepackage{hyperref}
\usepackage{caption}
\usepackage{tabularx}
\usepackage{enumitem}

\begin{document}

\section*{Дефиниции}

\section*{Задачи}
\subsection*{Задача 1.1 - Kenneth Rosen}
Съществува ли двуделен граф с нечетен брой върхове, който има Хамилтонов цикъл?

\subsection*{Лесни}
\subsubsection*{Задача 1.1 - Записки на Ангел Димитриев}
Да се докаже, че във всеки граф турнир има Хамилтонов маршрут.
\subsubsection*{Задача 1.2 - Записки на Ангел Димитриев}
Да се докаже, че за двуделен граф с дялове $A$ и $B$, който е хамилтонов е в сила, че $|A| = |B|$. \subsubsection*{Задача 1.3 - Записки на Ангел Димитриев}
Нека $G$ е грид-граф. Нека грида е $p \times q$. Да се докаже, че $G$ хамилтонов $\iff$ $p$ или $q$ е четно.



\subsection*{По-забавни}
\subsubsection*{Задача 2.1 - Теорема на Dirac}
Нека $G = (V, E)$ е граф, такъв че $n \geq 3$ и $\delta(G) \geq \lceil \frac{n}{2} \rceil$.
\subsubsection*{Задача 2.2 - Теорема на Ore}
Нека $G = (V, E)$ е граф, такъв че $n \geq 3$ и $\forall u, v \in V: d(u) + d(v) \geq n$.
\subsubsection*{Задача 2.3 - Kenneth Rosen}
Докажете, че има обход на коня на дъска $3 \times 4$.
\subsubsection*{Задача 2.4 - Kenneth Rosen}
Докажете, че няма обход на коня на дъска $3 \times 3$.
\subsubsection*{Задача 2.5 - Kenneth Rosen}
Имаме грид $11 \times 13$. Намираме се на клетка $(5, 3)$. Възможно ли е да обходим всички клетки и да се върнем обратно, откъдето започнахме?

\section*{Решения}

\end{document}