\documentclass[12pt]{article}
\usepackage{lingmacros}
\usepackage{tree-dvips}
\usepackage[utf8]{inputenc}
\usepackage[russian]{babel}
\usepackage{amsmath,amssymb}
\usepackage{multirow}
\usepackage{hyperref}
\usepackage{caption}
\usepackage{tabularx}
\usepackage{enumitem}

\begin{document}

\section*{Задачи}

\subsection*{Лесни}
\subsubsection*{Задача 1.1}
Да се докаже, че $\mathbb{N} \times \mathbb{N}$ е изброимо.
\subsubsection*{Задача 1.2 - записки на Ангел Димитриев}
Нека $S_{bool}$ е множеството на всички булеви вектори. Да се докаже, че $S_{bool}$ е изброимо.
\subsubsection*{Задача 1.3 - записки на Ангел Димитриев}
Да се докаже, че множеството от крайните редици от естествени числа е избороимо.
\subsection*{Задача 1.4 - Задачи за самоподготовка по Дискретни структури - 2020/2021 - Добромир Кралчев}
Дадени са множествата $K = \{ (a, b) \in \mathbb{R}^2 | \frac{a}{b} = 2 \}$ и $L = \{ (a, b) \in \mathbb{R}^2 | \frac{b}{a} = 3 \}$. Да се докаже, че $|K| = |L|$, тоест че има биекция между двете множества.

\subsection*{По-забавни}
\subsection*{Задача 2.1 - Задачи за самоподготовка по Дискретни структури - 2021/2022 - Добромир Кралчев}
\paragraph*{}
Едно семейство от множества се нарича верига относно релацията на включване, ако за всеки две различни множества $A, B$ от семейството е в сила включването $A \subset B$ или $B \subset A$. Постройте неизброима верига от подмножества на $\mathbb{N}$.
\paragraph*{}
\emph{Упътване: } Използвайте множеството $\mathbb{Q}$ като посредник. Както е известно, $\mathbb{Q}$ притежава две противоположни свойства:
\begin{itemize}
    \item $\mathbb{Q}$ е изброимо, следователно е равномощно на $\mathbb{N}$(в този смисъл $\mathbb{Q}$ е ``малко'' множество)
    \item $\mathbb{Q}$ е гъсто в $\mathbb{R}$, тоест между всеки две реални числа има поне едно рационално число (в този смисъл $\mathbb{Q}$ е ``голямо'' множество)
\end{itemize}
\subsection*{Задача 2.2 - домашна работа - КН - 2017}
Съществува ли множество от точки в тримерното пространство, което има

\begin{enumerate}[label=\Alph*)]
    \item поне една, но не повече от краен брой общи точки с всяка равнина?
    \item изброимо безкрайно много общи точки с всяка равнина?
\end{enumerate}
\emph{Забележка: } търсят се две отделни множества за двата въпроса, тъй като те са несъвместими.
\subsection*{Задача 2.3}
Да се докаже, че няма биекция между $S$ и $2^S$ за произволно множество $S$.

\section*{Решения}

\end{document}