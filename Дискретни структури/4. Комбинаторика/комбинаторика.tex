\documentclass[12pt]{article}
\usepackage{lingmacros}
\usepackage{tree-dvips}
\usepackage[utf8]{inputenc}
\usepackage[russian]{babel}
\usepackage{amsmath,amssymb}
\usepackage{multirow}
\usepackage{hyperref}
\usepackage{caption}
\usepackage{tabularx}
\usepackage{enumitem}

\begin{document}

\section*{Дефиниции}

\section*{Задачи}

\subsection*{Лесни}
\subsubsection*{Задача 1.1}
По колко начина могат от 5 студенти да се изберат 3-ма за първо, второ и трето място? 
\subsubsection*{Задача 1.2}
Колко пароли с пет символа могат да се измислят, ако се използват само малки латински букви и цифри?
\subsubsection*{Задача 1.3}
По колко начина от 12 учители може да се избере комисия от 6 души?
\subsubsection*{Задача 1.4 - записки на Ангел Димитриев}
Да се докаже, че $\binom{n+1}{k} = \displaystyle\sum_{p=0}^{n} \binom{n-p}{k-p}$
\subsubsection*{Задача 1.5}
Колко са булевите вектори с $n$ елемента, които имат точно $k$ нули.
\subsubsection*{Задача 1.6}
Имаме неограничено количество от топчета, които са $n$ цвята. По колко начина можем да изберем комплект от $k$ топчета като имаме право да повтаряме цветове и редът няма значение?
\subsubsection*{Задача 1.7 - записки на Ангел Димитриев}
Колко са булевите вектори с дължина $n$, които започват и завършват с различен знак?
\subsubsection*{Задача 1.8 - записки на Ангел Димитриев}
Колко са булевите вектори с дължина $n$, които имат четен брой единици?
\subsection*{Задача 1.9 - записки на Ангел Димитриев}
Колко са естествените числа в интервала $[1000, 9999]$, които започват с 3 или завършват с 4.
\subsection*{Задача 1.10 - записки на Ангел Димитриев}
Колко са естествените числа в интервала $[1000, 9999]$, които не започват с 3 и не завършват с 4.
\subsubsection*{Задача 1.11}
Колко са трицифрените числа, които не завършват на 3?
\subsubsection*{Задача 1.12 - записки на Ангел Димитриев}
Ако имаме 7 кашона всеки с различен строителен материал и 10-етажна сграда. По колко начина можем да наредим кашоните по етажите така, че на последния етаж да има поне два вида материали.
\subsubsection*{Задача 1.13}
Колко пермутации има на числата от 1 до $n$, $n \geq 3$ така, че числата 1, 2 и 3 да бъдат в този ред(може да не са последователни, но е важно да се срещат в тази последователност). 
\subsubsection*{Задача 1.14 - файл със задачи по комбинаторика}
Колко булеви вектори с дължина $n$ и $k$ единици има, при които няма две съседни единици?
\subsubsection*{Задача 1.15 - файл със задачи по комбинаторика}
Колко кръгови булеви вектори с дължина $n$ и $k$ единици има, при които няма две съседни единици(кръгов вектор е нормален вектор, при който първият и последният елемент са съседи)?
\subsection*{Задача 1.16}
Да се докаже, че има по-малко от $350$ прости числа по-малки или равни на $1000$. 
\subsection*{Задача 1.17}
Колко са естествените числа в интервала $[1, 100]$, които не се делят на 2, 3 и 8.
\subsection*{Задача 1.18}
Колко са решенията на уравнението $x_1 + x_2 + x_3 + x_4 = 100$, където $x_1, x_2, x_3, x_4 \in \mathbb{N}$?
\subsection*{Задача 1.19}
Колко са решенията на уравнението $x_1 + x_2 + x_3 + x_4 = 100$, където $x_1, x_2, x_3, x_4 \in \mathbb{N}$ и $x_1 \geq 2, x_2 \geq 10, x_3 \geq 1, x_4 > 2$?
\subsection*{Задача 1.20}
Колко е коефициентът пред $x^{222}y^{354}z^{424}$ в израза $(x + y + z)^{1000}$.
\subsection*{Задача 1.21 - файл със задачи по комбинаторика}
Задача 41
\subsubsection*{Задача 1.22 - задачи за самоподготовка по дискретни структури - 2021/2022 - Добромир Кралчев}
Да се докаже, че ако $m, n \in \mathbb{N}$ и $0 \leq m \leq n$, то
\begin{equation*}
    \displaystyle\sum_{k=m}^n \binom{k}{m} = \binom{n+1}{m+1}
\end{equation*}
\subsubsection*{Задача 1.23 - задачи за самоподготовка по дискретни структури - 2021/2022 - Добромир Кралчев}
В множеството $\{ \textup{а}, \textup{б}, \textup{в}, ..., \textup{ю}, \textup{я}, 0, 1, ..., 8, 9 \} $ съставено от тридесетте кирилски букви и десетте арабски цифри, е въведена частична строга наредба:
\begin{equation*}
    \textup{a} < \textup{б} < ... < \textup{я} ; 0 < 1 < ... < 9 
\end{equation*}
Тази наредба е непълна, защото всяка буква е несравнима с всяка цифра. По колко начина дадената непълна строга наредба може да се разшири до пълна строга наредба?

\subsubsection*{Задача 1.24}
Да се докаже, че $\binom{2n}{n}$ е четно за всяко $n \geq 1$.

\subsubsection*{Задача 1.25}
Да се докаже с еквивалентни преобразувания(плюс индукция) и с комбинаторни разсъждения, че $\displaystyle\sum_{k=0}^n {\binom{n}{k}}^2 = \binom{2n}{n}$.

\subsubsection*{Задача 1.25}
Да се докаже с еквивалентни преобразувания(плюс индукция) и с комбинаторни разсъждения, че $\displaystyle\sum_{k=0}^p \binom{n}{k}\binom{m}{p-k} = \binom{m+n}{p}$.

\subsection*{Задача 1.26 - Kenneth H. Rosen}
Да се докаже с комбинаторни разсъждения и с биномната формула на Нютон следното тъждество:
\begin{equation*}
    \displaystyle\sum_{k=0}^{n} {(-1)^k \binom{n}{k}} = 0
\end{equation*}
\subsection*{Задача 1.27 - Kenneth H. Rosen}
Да се докаже с комбинаторни разсъждения следното тъждество:
\begin{equation*}
    \binom{2n}{2} = 2 \binom{n}{2} + n^2
\end{equation*}
\subsection*{Задача 1.28 - Kenneth H. Rosen}
Да се докаже с комбинаторни разсъждения следното тъждество:
\begin{equation*}
    \displaystyle\sum_{k=1}^{n} {k \binom{n}{k}} = n 2^{n-1}
\end{equation*}
\textbf{Hint: } разгледайте броя начини да се избере комитет от $n$ души и след това за този комитет да се избере лидер.

\subsection*{По-забавни}
\subsubsection*{Задача 2.1}
Колко са решенията на уравнението $x_1 + x_2 + x_3 = 11$, където $x_1, x_2, x_3 \in \mathbb{N}$ и $x_1 \leq 3, x_2 \leq 4, x_3 \leq 6$?
\subsubsection*{Задача 2.2}
Колко е броя на деранжиментите на числата $1, 2, ..., n$(това са пермутациите, при които няма число което да си е на мястото, тоест число на позиция $k$ не може да е равно на $k$)?
\subsubsection*{Задача 2.3 - домашна работа - КН - 2021}
\paragraph*{}
$n$ души седят около кръгла маса. \emph{Дискомфортът между двама души} е абсолютната разлика на височините им. \emph{Общия дискомфорт} на масата е равен на
сумата на дискомфортите между хората седящи на съседни места.
\paragraph*{}
Нека означим височините на хората с $h_1, h_2, ..., h_n$, като $h_1 \leq h_2 \leq ... \leq h_n$. Колко е минималният общ дискомфорт, който може да се достигне при подходящо нареждане на хората около масата? Дайте пример за едно такова нареждане.
\paragraph*{}
С други думи, ако наредим хората с пермутацията $p_1, p_2, ..., p_n$, то общият дискомфорт ще бъде $\displaystyle\sum_{i=1}^{n-1} |h_{p_i} - h_{p_{i+1}}| + |h_{p_1} - h_{p_n}|$.

\subsection*{Задача 2.4 - задачи за самоподготовка по дискретни структури - 2021/2022 - Добромир Кралчев}
С помощта на цифрите 1, 2, 5, 8 и 9 са съставени всички възможни петцифрени числа с различни цифри. Намерете сбора и броя на тези числа.

\subsection*{Задача 2.5 - задачи за самоподготовка по дискретни структури - 2021/2022 - Добромир Кралчев}
С помощта на цифрите 1, 2, 3, 4, 5 и 6 са съставени всички възможни шестцифрени числа с различни цифри. Тези числа са подредени във възходящ ред(най-малкото число е първо, а най-голямото — последно). Кое число се намира на 423 -то място?

\subsection*{Задача 2.6 - изпит-задачи - КН - 2021}
Докажете с комбинаторни разсъждения, че за всеки цели положителни числа $m$ и $n$ е изпълнено:
\begin{equation*}
    \displaystyle\sum_{k=0}^{2n+2} {(-1)^k \binom{n}{k} (n-k)^m} = \begin{cases}
        0 & \textup{ако } m < n \\
        \frac{1}{\displaystyle\sum_{k=0}^n {\binom{n}{k}}^2} \cdot \frac{(2n)!}{n!} & \textup{ако } m = n
    \end{cases}
\end{equation*}

\subsection*{Задача 2.7 - задачи за самоподготовка по дискретни структури - 2017/2018 - Добромир Кралчев}
Докажете комбинаторното тъждество
\begin{equation*}
    \binom{2}{2} \binom{n}{2} + \binom{3}{2} \binom{n-1}{2} + \binom{4}{2}\binom{n-2}{2} + ... + \binom{n}{2} \binom{2}{2} = \binom{n+3}{5}
\end{equation*}
по три различни начина: с индукция, с двукратно броене и с биномната формула на Нютон.
$ \\ $\textbf{Тази е много гадна, особено частта биномната формула. Индукцията пък е твърде много смятане и не си заслужава.}

\section*{Решения}

\end{document}