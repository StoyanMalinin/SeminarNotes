\documentclass[12pt]{article}
\usepackage{lingmacros}
\usepackage{tree-dvips}
\usepackage[utf8]{inputenc}
\usepackage[russian]{babel}
\usepackage{amsmath,amssymb}
\usepackage{multirow}
\usepackage{hyperref}
\usepackage{caption}
\usepackage{tabularx}
\usepackage{enumitem}

\begin{document}

\section*{Дефиниции}

\section*{Задачи}

\subsection*{Лесни}
\subsubsection*{Задача 1.1}
Задача 19 от файла със задачи по комбинаторика.
\subsubsection*{Задача 1.2}
Задача 20 от файла със задачи по комбинаторика.
\subsubsection*{Задача 1.3}
Задача 21 от файла със задачи по комбинаторика.
\subsubsection*{Задача 1.4}
Задача 23 от файла със задачи по комбинаторика.
\subsubsection*{Задача 1.5}
Задача 24 от файла със задачи по комбинаторика.
\subsubsection*{Задача 1.5}
Задача 26 от файла със задачи по комбинаторика.
\subsection*{Задача 1.6 - Kenneth Rosen}
На една улица има 51 къщи, номерирани с числата от 1000 до 1099 включително. Да се докаже, че има поне две къщи с последователни номера. 
\subsection*{Задача 1.7 - Kenneth Rosen}
Нека $n_1, n_2, ..., n_t$ да бъдат положителни естествени числа. Покажете, че ако $n_1 + n_2 + ... + n_t + - t + 1$ се поставят във $t$ кутии, то за някое $i \in \{ 1, 2, ..., t \}$ $i$-тата кутия съдържа поне $n_i$ обекта.

\subsection*{По-забавни}
\subsubsection*{Задача 2.1}
Задача 22 от файла със задачи по комбинаторика.
\subsubsection*{Задача 2.2}
Задача 25 от файла със задачи по комбинаторика.


\section*{Решения}

\end{document}