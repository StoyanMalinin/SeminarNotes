\documentclass[12pt]{article}
\usepackage{lingmacros}
\usepackage{tree-dvips}
\usepackage[utf8]{inputenc}
\usepackage[russian]{babel}
\usepackage{amsmath,amssymb}
\usepackage{multirow}
\usepackage{hyperref}
\usepackage{caption}
\usepackage{tabularx}
\usepackage{enumitem}

\begin{document}

\section*{Дефиниции}

\section*{Задачи}

\subsection*{Лесни}
\subsubsection*{Задача 1.1}
Да се намери формула за редицата на фибоначи
$f_n = \begin{cases}
    0 & n = 0 \\
    1 & n = 1 \\
    f_{n-2} + f_{n-1} & n \geq 2 \\
\end{cases}$
\subsubsection*{Задача 1.2 - Добромир Кралчев, Александър Каракушев}
Да се реши рекурентното уравнение:
\begin{equation*}
    T(1) = 1, T(n) = \frac{1}{\frac{1}{T(n-1)} + n^2}
\end{equation*}
\subsubsection*{Задача 1.3}
Да се реши рекурентното уравнение: $ \\ $
$a_1 = 1$ $ \\ $
$a_n = a_{n-1} + n$ $ \\ $
Какво пресмята това уравнение?

\subsubsection*{Задача 1.4 - Kenneth H. Rosen}
Да се реши рекурентното уравнение:
\begin{equation*}
    a_n = \begin{cases}
        0 & n = 0 \\
        4 & n = 1 \\
        \frac{a_{n-2}}{4} & n \geq 2 \\
    \end{cases}
\end{equation*}
\subsubsection*{Задача 1.5 - Kenneth H. Rosen}
Да се реши рекурентното уравнение:
\begin{equation*}
    a_n = \begin{cases}
        4 & n = 0 \\
        10 & n = 1 \\
        6a_{n-1} - 8a_{n-2}  & n \geq 2 \\
    \end{cases}
\end{equation*}

\subsection*{Задача 1.6 - Kenneth H. Rosen}
Да се реши рекурентното уравнение $a_0 = 2$, $a_n = 2a_{n-1} + 2^n$.
\subsection*{Задача 1.6 - Kenneth H. Rosen}
Да се реши рекурентното уравнение $a_0 = 4$, $a_n = 2a_{n-1} + n + 5$.
\subsection*{Задача 1.7 - Kenneth H. Rosen}
Да се реши рекурентното уравнение 

\begin{equation*}
    a_0 = 1, a_1 = 4, a_n = 4a_{n-1} - 3a_{n-2} + 2^n + n + 3.
\end{equation*}

\subsection*{По-забавни}
\subsubsection*{Задача 2.1 - Добромир Кралчев}
Да се докаже, че числото $(4 + \sqrt[]{7})^{2015} + (4 - \sqrt[]{7})^{2015}$ е цяло и да се намери цифрата на единиците му. 
\subsubsection*{Задача 2.2 - Добромир Кралчев}
Да се реши рекурентното уравнение $a_1 = 87, a_{n+1} = \frac{a_n - 1}{a_n}$.
\subsubsection*{Задача 2.3 - семестриално контролно - КН - 2021}
Разгледайте рекурентното уравнение
\begin{equation*}
    a_n = \begin{cases}
        3, & \textup{ако } n = 0 \\
        na_{n-1} - (n-1), & \textup{ако } n > 0 \\ 
    \end{cases}
\end{equation*}
\begin{enumerate}
    \item Обяснете защо това уравнение не може да се реши чрез изучавания на лекции метод с характеристичното уравнение.
    \item Напишете стойностите на $n!$ за n = 0, 1, ..., 8.
    \item Напишете стойностите на $a_n$ за n = 0, 1, ..., 8.
    \item Открийте закономерност при стойностите на $a_n$. С други думи, отгатнете решение на рекурентното уравнение.
    \item Докажете по индукция, че отгатнатото от Вас решение на рекурентното уравнение наистина е решение.
\end{enumerate}
\subsubsection*{Задача 2.5 - Добромир Кралчев - приложение на рекурентните уравнения}
Колко $n$-цифрени цели положителни числа съдържат в десетичния си запис четен брой тройки (включително нито една)?
\subsubsection*{Задача 2.6 - Добромир Кралчев - приложение на рекурентните уравнения}
По колко начина числата 1, 2, 3, ..., $n$ могат да се наредят в редицата така, че всеки член (без първия) да се различава с единица от някое от числата вляво от него?


\section*{Решения}

\end{document}