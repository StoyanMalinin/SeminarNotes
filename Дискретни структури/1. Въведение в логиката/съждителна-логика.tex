\documentclass[12pt]{article}
\usepackage{lingmacros}
\usepackage{tree-dvips}
\usepackage[utf8]{inputenc}
\usepackage[russian]{babel}
\usepackage{amsmath,amssymb}
\usepackage{multirow}
\usepackage{hyperref}
\usepackage{caption}
\usepackage{tabularx}

\begin{document}

\section*{Задачи}

\subsection*{Лесни}
\subsubsection*{Задача 1.1}
Задача 14 от упражнението за съждителна логика.
\subsubsection*{Задача 1.2}
Напишете отрицанието на твърдението ``Ако пече слънце, то нося шапка.``
\subsubsection*{Задача 1.3}
Задача 18 от упражнението за съждителна логика.
\subsubsection*{Задача 1.4}
Задача 19 от упражнението за съждителна логика.
\subsubsection*{Задача 1.5}
Преобразувайте следното твърдение без да използвате \emph{или}: ``Вали дъжд или не нося чадър``.
\subsubsection*{Задача 1.6}
Задача 3 от упражнението за съждителна логика.
\subsubsection*{Задача 1.7 - семестриално контролно 2021}
Докажете следните еквивалентности чрез еквивалентни преобразувания
\begin{itemize}
    \item $(p \rightarrow q) \land (p \rightarrow r) \equiv p \rightarrow (q \land r)$
    \item $(p \rightarrow r) \land (q \rightarrow r) \equiv (p \lor q) \rightarrow r$
    \item $(p \rightarrow q) \lor (p \rightarrow r) \equiv p \rightarrow (q \lor r)$
    \item $(p \rightarrow r) \lor (q \rightarrow r) \equiv (p \land q) \rightarrow r$
\end{itemize}
\subsubsection*{Задача 1.8 - Kenneth H. Rosen}
Да се докаже, че $\neg p \rightarrow (q \rightarrow r) \equiv q \rightarrow (p \lor r)$.
\subsubsection*{Задача 1.9 - Kenneth H. Rosen}
Да се докаже, че $(p \rightarrow q) \rightarrow r$ и $p \rightarrow (q \rightarrow r)$ не са логически еквивалентни.
\subsubsection*{Задача 1.10 - семестриално контролно - КН - 2016}
Какво представлява съждителният израз $(\neg p \land q) \land (q \leftrightarrow r) \rightarrow (p \rightarrow r)$: тавтология, противоречие или условност

\subsection*{По-забавни}
\subsubsection*{Задача 2.1}
Задача 11 от упражнението за съждителна логика.
\subsection*{Задача 2.2 - изпит на специалност Информатика - 2018}
\paragraph*{}
Две от следните твърдения са еквивалентни. Намерете кои са те и обяснете защо не са еквивалентни на другото твърдение. Обосновете формално и подборно отговорите си, използвайки съждителна логика.
\begin{enumerate}
    \item Ако грее слънце, то уча и тренирам.
    \item Не грее слънце или уча и тренирам.
    \item Ако уча, то тренирам и грее слънце.
\end{enumerate}
\subsubsection*{Задача 2.3 - Kenneth H. Rosen}
Намерете съставно съждение с променливите $p$, $q$ и $r$, което е истина тогава и само тогава, когато точно две от променливите са истина и една е лъжа.
$ \\ $ \emph{Hint: }Направете го от дизюнкция от конюнкции. Включете конюнкция за всяка комбинация на променливите, за която твърдението трябва да е истина.
$ \\ $ \emph{Вниамние: } малко няма смисъл за сега от тази задача, понеже тя е просто теоремата на Бул.
\subsubsection*{Задача 2.4 - Kenneth H. Rosen}
Намерете съставно съждение с променливите $p$, $q$ и $r$, което е истина тогава и само тогава, когато $p$ и $q$ са истина и $r$ е лъжа.
$ \\ $ \emph{Hint: } Използвайте конюнкция от от всяка променлива или нейната негация.


\section*{Решения}

\end{document}