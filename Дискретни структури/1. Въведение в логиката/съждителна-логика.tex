\documentclass[12pt]{article}
\usepackage{lingmacros}
\usepackage{tree-dvips}
\usepackage[utf8]{inputenc}
\usepackage[russian]{babel}
\usepackage{amsmath,amssymb}
\usepackage{multirow}
\usepackage{hyperref}
\usepackage{caption}
\usepackage{tabularx}

\begin{document}

\section*{Задачи}

\subsection*{Лесни}
\subsubsection*{Задача 1.1}
Задача 14 от упражнението за съждителна логика.
\subsubsection*{Задача 1.2}
Напишете отрицанието на твърдението ``Ако пече слънце, то нося шапка.``
\subsubsection*{Задача 1.3}
Задача 18 от упражнението за съждителна логика.
\subsubsection*{Задача 1.4}
Задача 19 от упражнението за съждителна логика.
\subsubsection*{Задача 1.5}
Преобразувайте следното твърдение без да използвате \emph{или}: ``Вали дъжд или не нося чадър``.
\subsubsection*{Задача 1.6}
Задача 3 от упражнението за съждителна логика.
\subsubsection*{Задача 1.7 - семестриално контролно 2021}
Докажете следните еквивалентности чрез еквивалентни преобразувания
\begin{itemize}
    \item $(p \rightarrow q) \land (p \rightarrow r) \equiv p \rightarrow (q \land r)$
    \item $(p \rightarrow r) \land (q \rightarrow r) \equiv (p \lor q) \rightarrow r$
    \item $(p \rightarrow q) \lor (p \rightarrow r) \equiv p \rightarrow (q \lor r)$
    \item $(p \rightarrow r) \lor (q \rightarrow r) \equiv (p \land q) \rightarrow r$
\end{itemize}

\subsection*{По-забавни}
\subsubsection*{Задача 2.1}
Задача 11 от упражнението за съждителна логика.

\section*{Решения}

\end{document}