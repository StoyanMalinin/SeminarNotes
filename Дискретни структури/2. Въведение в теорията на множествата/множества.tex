\documentclass[12pt]{article}
\usepackage{lingmacros}
\usepackage{tree-dvips}
\usepackage[utf8]{inputenc}
\usepackage[russian]{babel}
\usepackage{amsmath,amssymb}
\usepackage{multirow}
\usepackage{hyperref}
\usepackage{caption}
\usepackage{tabularx}

\begin{document}

\section*{Задачи}

\subsection*{Лесни}
\subsubsection*{Задача 1.1}
Задачи 9 и 10 от упражнението за множества.

\subsubsection*{Задача 1.2}
Задача 11 от упражнението за множества.

\subsubsection*{Задача 1.3}
Задача 12 от упражнението за множества.

\subsubsection*{Задача 1.4 - първо контролно на специалност информатика 2019/2020}
Нека $A, B, C, D \subseteq X$. Докажете или опровергайте, че ако $\forall x \in X (x \in A \rightarrow x \in C \land x \in B)$, то $(B \cup C) \setminus B = \overline{\overline{C} \cap \overline{A}} \cap \overline{B}$

\subsubsection*{Задача 1.5}
Задача 25 от упражнението за множества.

\subsubsection*{Задача 1.6}
Задача 28 от упражнението за множества.

\subsubsection*{Задача 1.7}
Задача 29 от упражнението за множества.

\subsubsection*{Задача 1.8}
Докажете, че ако $C \cap B = \emptyset$, то $(A \triangle B) \cup C = (A \cup C) \triangle B$.

\subsubsection*{Задача 1.9 - контролно на специалност компютърни науки 2016}
Разгледайте следните твърдения и формулирайте всяко твърдение на езика на предикатната логика. Образувайте и отрицанията на двете твърдения, като при това никъде във формулировката да не се среща знакът $\neg$.
\begin{itemize}
    \item Всяко цяло число, кратно на 4, може да се представи като сума от квадратите на две цели числа.
    \item За всяко реално число $x \geq -1$ и за всяко естествено число $n$ е в сила неравенството $(1+x)^n \geq 1 + nx$.  
\end{itemize}

\subsubsection*{Задача 1.10 - Kenneth H. Rosen}
Изкажете тези твърдения, използвайки предоставените предикати.
\begin{enumerate}
    \item Всички лъвове са свирепи.
    \item Някои лъвове не пият кафе.
    \item Някои свирепи животни не пият кафе.
\end{enumerate}
\begin{enumerate}
    \item $P(x) \iff$ ``$x$ е лъв''
    \item $Q(x) \iff$ ``$x$ е свирепо''
    \item $R(x) \iff$ ``$x$ пие кафе''
\end{enumerate} 
Нека също за пълнота множеството на всички животни да бъде $A$. Нека също и домейна на трите предиката да бъде $A$.

\subsubsection*{Задача 1.11 - Задачи за самоподготовка по Дискретни структури - 2021/2022 - Добромир Кралчев}
Да се докаже, че ако $A \subseteq B$, то $((C \cup A) \setminus B) \cap A = \emptyset$

\subsection*{По-забавни}

\subsection*{Задача 2.1 - семестриално контролно на специалност компютърни науки 2016}
Намерете множеството $X$ от системата като го изразите чрез множествата $A, B, C$ с помощта на сечение, обединение и разлика.

$$ \begin{cases}
    C \cup X = (B \setminus A) \cup C \\
    C \cap X =  (A \cup B) \cap C \\
\end{cases} $$

\subsection*{Задача 2.2 - Задачи за самоподготовка по Дискретни структури - 2021/2022 - Добромир Кралчев}
Преценете дали е вярно следното твърдение:
\begin{equation*}
    (\forall A \in 2^{\mathbb{N}} \setminus \{ \emptyset \})(\exists x \in A)(x \textup{ е четно} \implies (\forall y \in A)(y \textup{ е четно}))
\end{equation*}

\section*{Решения}

\subsection*{Задача 2.1}
\paragraph*{}
Задачата има много по-красиво и адекватно решение от представеното тук. То може да се намери заедно с контролното. Това тук е с демонстрационни цели.
\paragraph*{}
Трябва да определим еднозначно за всеки елемент $x$ от нашия свят дали е елемент на $X$ или не. Ще разбием този въпрос на $2^3 = 8$ случая според това дали $x$ принадлежи на някое от $A, B, C$
\subsubsection*{Случай 0 - $x \not\in A, x \not\in B, x \not\in C$}
Допускаме, че $x \in X$. Тогава $x \in C \cup X$, но $x \not\in (B \setminus A) \cup C$, което е в противоречие с първото уравнение от условието. Следователно $x \not\in X$. Следователно $X \subseteq (A \cup B \cup C)$. 
\subsubsection*{Случай 1 - $x \not\in A, x \not\in B, x \in C$}
Допускаме, че $x \in X$. Тогава $x \in C \cap X$, но $x \not\in (A \cup B)$, следователно $x \not\in (A \cup B) \cap C$, което е противоречие с второто уравнение от условието. Следователно $x \not\in X$. Следователно $(C \setminus (A \cup B)) \cap X = \emptyset$.
\subsubsection*{Случай 2 - $x \not\in A, x \in B, x \not\in C$}
Допускаме, че $x \not\in X$. Следователно $x \not\in C \cup X$, но $x \in B \setminus A$ и $x \in (B \setminus A) \cup C$, което е противоречие. Следователно $x \in X$, следователно $B \setminus (A \cup C) \subseteq X$.
\subsubsection*{Случай 3 - $x \not\in A, x \in B, x \in C$}
Допускаме, че $x \not\in X$. Следователно $x \not\in X \cap C$, но $x \in (A \cup B)$ и $x \in (A \cup B) \cap C$, което е противоречие. Следователно $x \in X$. Следователно $(B \cap C) \setminus A \subseteq X$.
\subsubsection*{Случай 4 - $x \in A, x \not\in B, x \not\in C$}
Допускаме, че $x \in X$. Тогава $x \in C \cup X$, но $x \not\in (B \setminus A)$ и $x \not\in (B \setminus A) \cup C$, което е противоречие. Следователно $x \not\in X$. Следователно $(A \setminus (B \cup C)) \cap X = \emptyset$.
\subsubsection*{Случай 5 - $x \in A, x \not\in B, x \in C$}
Допускаме, че $x \not\in X$. Тогава $x \not\in C \cap X$, но $x \in (A \cup B)$ и $x \in (A \cup B) \cap C$, което е противоречие. Следователно $x \in X$. Следователно $(A \cap C) \setminus B \subseteq X$.
\subsubsection*{Случай 6 - $x \in A, x \in B, x \not\in C$}
Допускаме, че $x \in X$. Следователно $x \in C \cup X$, но $x \not\in (B \setminus A)$ и $x \not\in (B \setminus A) \not\in C$, което е противоречие. Следователно $x \not\in X$, следователно $((A \cap B) \setminus C) \cap X = \emptyset$.
\subsubsection*{Случай 7 - $x \in A, x \in B, x \in C$}
Допускаме, че $x \not\in X$. Тогава $x \not\in C \cap X$, но $x \in (A \cup B)$ и $x \in (A \cup B) \cap C$, което е противоречие. Следователно $x \in X$, следователно $A \cap B \cap C \subseteq X$.

\paragraph*{}
Разглеждайки всички тези случаи и знаейки, че $X \subseteq A \cup B \cup C$, можем да заключим, че $X = (B \setminus (A \cup C)) \cup ((B \cap C) \setminus A) \cup ((A \cap C) \setminus B) \cup (A \cap B \cap C)$.

\end{document}